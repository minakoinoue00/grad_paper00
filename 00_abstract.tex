% ■ アブストラクトの出力 ■
%	◆書式:
%		begin{jabstract}〜end{jabstract}	:日本語のアブストラクト
%		begin{eabstract}〜end{eabstract}	:英語のアブストラクト
%		※ 不要ならばコマンドごと消せば出力されない。



% 日本語のアブストラクト
\begin{jabstract}

	近年, 国内旅行をする人数が減少しているというデータがある. 全盛期であった19ーー年代と比べて, 近年特に2010年代はーー%減少している. また, 日本の温泉旅館もその数を
	減らしている. 1998年にはーー件あった温泉旅館も2013年にはーー件にまで減少している. 旅館経営自体も1998年から2013年の間にーー件(ーー%)の減少となっている. 
	もちろん, 単純に国内旅行をする人が減っているわけではない. 近年, 海外旅行をする人数は増加傾向にある. 

	星野リゾートというのは, 1904年に長野県軽井沢町に創立された星野温泉を現社長星野佳路氏によって抜本的な改革が行われリゾート再生企業として今注目を浴びている企業である. 

テンプレートの説明を、テンプレート自身を使って説明する。これは @kurokobo による卒業論文のための\LaTeX テンプレートを修士論文用に改造し、さらにUTF-8化やMakefile等の添付をしたものである。

この部分には一般には論文のアブストラクトを書く。日本語のアブストラクトを書きたいなら、\verb|\begin{jabstract}| と \verb|\end{jabstract}| の間に文章を書けば、今のこのページのように体裁が勝手に整って出力される。英語のアブストラクトは \verb|\begin{eabstract}| と \verb|\end{eabstract}| の間に書けば、次ページのような体裁で出力される。

両方を書けば、日本語と英語の両方のアブストラクトが並んで出力される(この文書はサンブルなので両方書いてある)。ページ順序は、コマンドを書いた順序の通り。どちらか一方のみを出力したい場合は、不要な方をコマンド自体を含め削除する。

このあたりの詳細もあとで書く。基本的には、{\tt main.tex}を上から順にいじっていけばできるはず。

\end{jabstract}



% 英語のアブストラクト
%\begin{eabstract}

%Eigo ga dekinai node Roma-ji de soreppoi hunniki wo daseruto iina.

%Murippoi desu ne.

%Write down your abstract here. Write down your abstract here. Write down your abstract here. Write down your abstract here. Write down your abstract here. Write down your abstract here.

% Write down your abstract here. Write down your abstract here. Write down your abstract here. Write down your abstract here. Write down your abstract here. Write down your abstract here. Write down your abstract here. Write down your abstract here. Write down your abstract here. Write down your abstract here. Write down your abstract here. Write down your abstract here.
 
%Write down your abstract here. Write down your abstract here.

%\end{eabstract}
